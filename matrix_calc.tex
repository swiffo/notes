\documentclass{article}

\usepackage{amsmath}
\usepackage{amssymb}
\usepackage{amsthm}

\theoremstyle{plain}
\newtheorem{theorem}{Theorem}[section]
\newtheorem{lemma}[theorem]{Lemma}
\newtheorem{proposition}[theorem]{Proposition}
\newtheorem{corollary}[theorem]{Corollary}
\newtheorem{definition}[theorem]{Definition}
\newtheorem{remark}[theorem]{Remark}
\newtheorem{notation}[theorem]{Notation}

\newcommand{\diff}[2]{\frac{\partial {#1}}{\partial {#2}}}
\newcommand{\R}{\mathbb{R}}

\title{Matrix and Vector Differentiation}
\author{Rolf D. Svegstrup}

\begin{document}
\maketitle

\section{Vector Differentiation}
\begin{definition}
Let $y = \psi(x) \in \R^m$ where $x$ ranges over $R^n$. Then $\diff{y}{x}$ is an $m \times n$-matrix given by

\begin{equation}
\diff{y}{x} = 
\begin{pmatrix} 
\diff{y_1}{x_1} & \cdots & \diff{y_1}{x_n} \\ 
\vdots && \vdots \\ 
\diff{y_n}{x_1} & \cdots & \diff{y_n}{x_m} 
\end{pmatrix}
\end{equation}
The matrix is called the \emph{Jacobian} of the transformation $\psi$.
\end{definition}

\begin{remark}
Note that if vectors $z = z(y)$ and $y = y(x)$, then $\diff{z}{x} = \diff{z}{y} \diff{y}{x}$. Also note that $\diff{x}{x} = I$.
\end{remark}

\begin{notation}
Throughout this section it will be assumed that $x$ is a vector variable, $y = y(x)$ and $z = y(x)$ are vectors depending on $x$ and that $A$ is a matrix not depending on $x$. Dimensions of $x$, $y$, $z$ and $A$ are such that the equations make sense.
\end{notation}

\begin{proposition}
\begin{equation}
\diff{A y}{x} = A \diff{y}{x}
\end{equation}
\end{proposition}
\begin{proof}
\begin{equation*}
\left(\diff{A y}{x}\right)_{ij} = \diff{(\sum_k A_{ik} y_k)}{x_j} = \sum_k A_{ik} \left(\diff{y}{x}\right)_{kj} = \left(A \diff{y}{x}\right)_{ij}.
\end{equation*}
\end{proof}

\begin{proposition}
\begin{equation}
\diff{y^t A z}{x} = z^t A^t \diff{y}{x} + y^t A \diff{z}{x}.
\end{equation}
\end{proposition}
\begin{proof}
\begin{equation*}
\begin{split}
\left( \diff{y^t A z}{x}\right)_{1j} &= \diff{\sum_{k, l} y_k A_{kl} z_l}{x_j} \\
	&= \sum_k  \diff{y_k}{x_j} (A z)_k + \sum_l (y^t A)_{1l} \diff{z_l}{x_j} \\
	&= \sum_k \left(\diff{y}{x}\right)_{kj} (A z)_k + \sum_l (y^t A)_{1l} \left(\diff{z}{x}\right)_{lj} \\
	&= \left( \left( \left( \diff{y}{x} \right)^t A z \right)^t+ y^t A \diff{z}{x} \right)_{1j}
\end{split}
\end{equation*}
\end{proof}

\begin{corollary}
Some simple corollaries follow:
\begin{enumerate}
\item $\diff{A x}{x} = A$.
\item $\diff{x^t A x}{x} = x^t(A + A^t)$.
\item $\diff{x^t x}{x} = 2 x^t$.
\item $\diff{y^t z}{x} = z^t \diff{y}{x} + y^t \diff{z}{x}$.
\end{enumerate}
\end{corollary}


\section{Differentiating Matrices by Scalars}

\begin{definition}
Let $A$ be an $m \times n$ matrix whose elements are functions of the scalar $\alpha$. Then
\begin{equation}
\diff{A}{\alpha} = \begin{pmatrix}
\diff{A_{11}}{\alpha} & \cdots & \diff{A_{1n}}{\alpha} \\
\vdots & & \vdots \\
\diff{A_{m1}}{\alpha} & \cdots & \diff{A_{mn}}{\alpha}
\end{pmatrix}.
\end{equation}
\end{definition}

\begin{lemma}
\begin{equation}
\diff{AB}{\alpha} = \diff{A}{\alpha} B + A \diff{B}{\alpha}.
\end{equation}
\end{lemma}

\begin{proposition}
Let $A$ be an invertible matrix. Then
\begin{equation}
\diff{A^{-1}}{\alpha} = - A^{-1} \diff{A}{\alpha} A^{-1}.
\end{equation}
\end{proposition}
\begin{proof}
\begin{equation*}
0 = \diff{A^{-1} A}{\alpha} = \diff{A^{-1}}{\alpha} A + A^{-1} \diff{A}{\alpha}.
\end{equation*}
Isolating $\diff{A^{-1}}{\alpha}$ completes the proof.
\end{proof}


\section{Applications}
\subsection{Linear Regression Coefficients}
Let there be given $N$ observations $(x_i, y_i)$ where $x_i \in \R^n$ and $y_i \in \R$. Let $y = (y_i) \in \R^N$. Let $X \in \R^{N \times (n+1)}$ be the matrix in which the $i$-th row is $1$ followed by $x_i$ as a row vector. We seek a coefficient vector $\beta \in \R^{n+1}$ such that $\lvert\lvert X \beta - y \rvert\rvert_2$ is minimized.

We find the $\beta$ minimizing the square of the norm using vector differentiation:
\begin{equation}
0 = \diff{}{\beta} (X \beta - y)^t (X \beta - y) = 2 (X \beta - y)^t X.
\end{equation}
Hence, $\beta^t X^t X = y^t X$. Assuming that $X^t X$ is invertible (columns of X being linearly independent), we find
\begin{equation}
\beta = (X^t X)^{-1} X^t y.
\end{equation}

\end{document}