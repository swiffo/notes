\documentclass{article}

\title{Probability Distributions}
\author{Rolf D. Svegstrup}

\usepackage{amsmath}

\begin{document}
\maketitle

\section{Bernoulli Distribution}
A \emph{Bernoulli random variable}, $X$, takes the value 1 with probability $p$ and 0 otherwise. Thus,
\begin{equation}
\label{Eq: Bernoulli properties}
\begin{split}
E(X) &= p \\ 
V(X) &= p(1-p).
\end{split}
\end{equation}

\section{Binomial Distribution}
A \emph{Binomial random variable}, $X$, is a sum of $n$ iid. Bernoulli random variables. We write $X \sim B(n, p)$. Trivially,
\begin{equation}
P(X = k) = {n \choose k} p^k (1-p)^{n-k}.
\end{equation}
From (\ref{Eq: Bernoulli properties}) it follows that
\begin{equation}
\begin{split}
E(X) &= np \\
V(X) &= np(1-p) .
\end{split}
\end{equation}

\end{document}